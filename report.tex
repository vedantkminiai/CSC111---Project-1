\documentclass[11pt]{article}
\usepackage{amsmath}
\usepackage{amsfonts}
\usepackage{amsthm}
\usepackage[utf8]{inputenc}
\usepackage[margin=0.75in]{geometry}

\title{CSC111 Winter 2026 Project 1}
\author{Lucas Hui, Vedant Kansara}
\date{\today}

\begin{document}
\maketitle

\section*{Running the game}
We should be able to run your game by simply running \texttt{adventure.py}. If you have any other requirements (e.g., installing certain modules), describe them here. Otherwise, skip this section.

\section*{Game Map}
Example game map below (edit it to show your actual game map):

\begin{verbatim}
-1   2  -1  -1
-1   3   6  -1
-1      -1  -1
 1   4   7   8
-1   5  -1  -1
\end{verbatim}

Starting location is: 1

IMPORTANT NOTE: There is a empty space because there is no path that goes from location 7 to 6 and 6 to 7 in the game data so we added a barrier to separate the two location.

\section*{Game solution}
List of commands:

["go east", "go south", "take usb drive", "go north", "go east", "go east", "take lucky mug","go west", "go west", "go north", "go east", "take laptop charger", "go west", "go south","go west", "deposit usb drive", "deposit laptop charger","deposit lucky mug"]

\section*{Lose condition(s)}
Description of how to lose the game:

The player loses the game if they run out of moves. The game ends immediately when the player reaches the maximum number of allowed moves (20) without successfully returning all required items to the dorm.

List of commands:

["go east", "go south", "go north", "go east", "go east", "go west", "go west", "go north", "go north","go south", "go east", "go west", "go south", "go south", "go north", "go west", "go east", "go west","go east", "go west"]

Which parts of your code are involved in this functionality:

The lose condition is implemented using the moves attribute in the AdventureGame class, which tracks how many actions the player has taken and is updated throughout gameplay. The value of moves is incremented whenever the player performs an action that costs a move, such as moving between locations, taking or depositing an item, and is compared against a fixed maximum defined by $MAX\_MOVE = 20$. At the end of each iteration of the main game loop, the program checks whether the move limit has been reached and ends the game with a loss if so.

% Copy-paste the above if you have multiple lose conditions and describe each possible way to lose the game

\section*{Inventory}

\begin{enumerate}
\item All location IDs that involve items in the game:

\item Item data:
\begin{enumerate}
    \item For Item 1:
    \begin{itemize}
    \item Item name: usb drive
    \item Item start location ID: 5
    \item Item target location ID: 1
    \end{itemize}
        \item For Item 2:
    \begin{itemize}
    \item Item name: laptop charger
    \item Item start location ID: 6
    \item Item target location ID: 1
    \end{itemize}
        \item For Item 3:
    \begin{itemize}
    \item Item name: lucky mug
    \item Item start location ID: 8
    \item Item target location ID: 1
    \end{itemize}
        \item For Item 4:
    \begin{itemize}
    \item Item name: toonie
    \item Item start location ID: 3
    \item Item target location ID: 2
    \end{itemize}
    % Copy-paste the above if you have more items, to list ALL items
\end{enumerate}

    \item Exact command(s) that should be used to pick up an item (choose any one or more items for this example), and the command(s) used to use/drop the item (can copy the list you assigned to \texttt{inventory\_demo} in the \texttt{simulation.py} file)

    \begin{enumerate}
        To pick up an item, the player must type "take" followed by the item’s name, such as "take usb drive", "take laptop charger", or "take lucky mug".

        To use or drop an item, the player must type "deposit" followed by the item’s name, such as "deposit usb drive", "deposit laptop charger", or "deposit lucky mug", and the deposit must occur at the item’s designated target location.
    \end{enumerate}

    \item Which parts of your code (file, class, function/method) are involved in handling the \texttt{inventory} command:

    \begin{enumerate}
        The player’s items are stored in the inventory attribute of the AdventureGame class. The inventory command is processed in the main game loop (inside the if choice in menu block), where the code iterates through game.inventory and prints each item’s name.
    \end{enumerate}
\end{enumerate}

\section*{Score}
\begin{enumerate}

    \item Briefly describe the way players can earn score in your game. Include the first location in which they can increase their score, and the exact list of command(s) leading up to the score increase:

    \begin{enumerate}
        Players earn score by depositing items at their correct target location using the command deposit "item name", which adds that item’s target\_points to game.score. The first location where the player can increase their score is Location 1 (Dorm), because the player can bring back an important item like the usb drive and deposit it there for points.
    \end{enumerate}

    \item Copy the list you assigned to \texttt{scores\_demo} in the \texttt{simulation.py} file into this section of the report:

    \begin{enumerate}
        ["go east", "go south", "take usb drive", "go north", "go west", "deposit usb drive"]
    \end{enumerate}

    \item Which parts of your code (file, class, function/method) are involved in handling the \texttt{score} functionality:

    \begin{enumerate}
        Score functionality is handled in AdventureGame class, mainly through the score attribute and the deposit\_item method (which checks the target location and increases self.score by itm.target\_points). The score is displayed to the player in the main game loop under the menu command score, where it prints game.score
    \end{enumerate}
\end{enumerate}

\section*{Enhancements}
\begin{enumerate}
    \item Describe your enhancement \#1 here
    \begin{itemize}
        \item Brief description of what the enhancement is (if it's a puzzle, also describe what steps the player must take to solve it):

        \begin{itemize}
            The game includes an undo command that allows the player to undo the most recent non-menu action, such as moving between locations, taking an item, or depositing an item. When an undo is successful, the game restores the previous location or item state, updates the inventory and location item lists if needed, reverses score changes for deposits, decreases the move count, and removes the most recent event from the event log. The player is limited to a fixed number of undo attempts (undo\_chances = 3), and undo does nothing if the most recent action was a menu command like look, inventory, score, or log.
        \end{itemize}
        \item Complexity level (choose from low/medium/high):   Medium
        \item Reasons you believe this is the complexity level (e.g., mention implementation details, how much code did you have to add/change from the baseline, what challenges did you face, etc.)

        \begin{itemize}
            This enhancement required modifying multiple parts of the program but did not require a completely new game system. It builds on the existing event log structure to determine the most recent action and introduces helper methods to reverse movement and item actions. The main challenge was ensuring consistency across inventory, score, move count, and event history, while also preventing undo from affecting menu commands. Since it reuses some existing structures, it keeps it at a medium level of complexity.
        \end{itemize}

        \item Name the parts of the code which are involved in this enhancement

        \begin{itemize}
            AdventureGame.undo\_action
            \newline
            AdventureGame.\_undo\_take\_item
            \newline
            AdventureGame.\_undo\_deposit\_item
        \end{itemize}

        \item Copy the list you assigned to \texttt{enhancements\_demo} in the \texttt{simulation.py} file into this section of the report:

        \begin{itemize}
            ["go east", "go north", "undo", "go north", "take toonie", "undo", "take toonie"] \newline\newline\newline\newline\newline
        \end{itemize}

    \end{itemize}

    \item Describe your enhancement #2 here
    \begin{itemize}
        \item Basic description of what the enhancement is:

        \begin{itemize}
            Some locations in the game contain a hangman-style word puzzle (stored in that location’s puzzle\_words list). If the player tries to take an item at one of these puzzle locations, the game first starts the hangman puzzle and the player must guess the hidden word one letter at a time within 10 tries. If the player solves the puzzle, they are allowed to take the item; if they fail, they cannot take the item (and they may choose to retry or quit the puzzle)
        \end{itemize}
        \item Complexity level (low/medium/high): Medium
        \item Reasons you believe this is the complexity level (e.g., mention implementation details)
        \begin{itemize}
            This enhancement adds a mini hangman game with its own loop and input validation, including random word selection from a list, tracking guesses, limiting tries, and checking for success or failure. It required multiple helper functions and modifying the take command so that the puzzle runs automatically when the player attempts to take an item at a puzzle location.
        \end{itemize}
        \item Name the parts of the code which are involved in this enhancement
        \begin{itemize}
            AdventureGame.puzzle
            \newline
            AdventureGame.generate\_word
            \newline
            AdventureGame.\_get\_valid\_guess
            \newline
            AdventureGame.\_apply\_guess
            \newline
            AdventureGame.update\_word
        \end{itemize}
        \item Copy the list you assigned to \texttt{enhancements\_demo} in the \texttt{simulation.py} file into this section of the report:
        \begin{itemize}
            ["go east", "go north", "take toonie"]
            \newline Notes for the puzzle:
            \newline
            After entering the "take toonie" command the hangman puzzle will appear and the words the player has to solve is either "printer", "bookshelf", or "file" (Each location has a different word list) by entering one letter at a time.
            \newline
            For example for the word 'file' the letter you have to input are ['f', 'i', 'l', 'e']
        \end{itemize}
    \end{itemize}
\end{enumerate}


\end{document}
